\documentclass[]{article}
\usepackage{lmodern}
\usepackage{amssymb,amsmath}
\usepackage{ifxetex,ifluatex}
\usepackage{fixltx2e} % provides \textsubscript
\ifnum 0\ifxetex 1\fi\ifluatex 1\fi=0 % if pdftex
  \usepackage[T1]{fontenc}
  \usepackage[utf8]{inputenc}
\else % if luatex or xelatex
  \ifxetex
    \usepackage{mathspec}
  \else
    \usepackage{fontspec}
  \fi
  \defaultfontfeatures{Ligatures=TeX,Scale=MatchLowercase}
\fi
% use upquote if available, for straight quotes in verbatim environments
\IfFileExists{upquote.sty}{\usepackage{upquote}}{}
% use microtype if available
\IfFileExists{microtype.sty}{%
\usepackage{microtype}
\UseMicrotypeSet[protrusion]{basicmath} % disable protrusion for tt fonts
}{}
\usepackage[margin=1in]{geometry}
\usepackage{hyperref}
\hypersetup{unicode=true,
            pdftitle={Data-tidying-lesson},
            pdfauthor={Amanda B. Young},
            pdfborder={0 0 0},
            breaklinks=true}
\urlstyle{same}  % don't use monospace font for urls
\usepackage{color}
\usepackage{fancyvrb}
\newcommand{\VerbBar}{|}
\newcommand{\VERB}{\Verb[commandchars=\\\{\}]}
\DefineVerbatimEnvironment{Highlighting}{Verbatim}{commandchars=\\\{\}}
% Add ',fontsize=\small' for more characters per line
\usepackage{framed}
\definecolor{shadecolor}{RGB}{248,248,248}
\newenvironment{Shaded}{\begin{snugshade}}{\end{snugshade}}
\newcommand{\AlertTok}[1]{\textcolor[rgb]{0.94,0.16,0.16}{#1}}
\newcommand{\AnnotationTok}[1]{\textcolor[rgb]{0.56,0.35,0.01}{\textbf{\textit{#1}}}}
\newcommand{\AttributeTok}[1]{\textcolor[rgb]{0.77,0.63,0.00}{#1}}
\newcommand{\BaseNTok}[1]{\textcolor[rgb]{0.00,0.00,0.81}{#1}}
\newcommand{\BuiltInTok}[1]{#1}
\newcommand{\CharTok}[1]{\textcolor[rgb]{0.31,0.60,0.02}{#1}}
\newcommand{\CommentTok}[1]{\textcolor[rgb]{0.56,0.35,0.01}{\textit{#1}}}
\newcommand{\CommentVarTok}[1]{\textcolor[rgb]{0.56,0.35,0.01}{\textbf{\textit{#1}}}}
\newcommand{\ConstantTok}[1]{\textcolor[rgb]{0.00,0.00,0.00}{#1}}
\newcommand{\ControlFlowTok}[1]{\textcolor[rgb]{0.13,0.29,0.53}{\textbf{#1}}}
\newcommand{\DataTypeTok}[1]{\textcolor[rgb]{0.13,0.29,0.53}{#1}}
\newcommand{\DecValTok}[1]{\textcolor[rgb]{0.00,0.00,0.81}{#1}}
\newcommand{\DocumentationTok}[1]{\textcolor[rgb]{0.56,0.35,0.01}{\textbf{\textit{#1}}}}
\newcommand{\ErrorTok}[1]{\textcolor[rgb]{0.64,0.00,0.00}{\textbf{#1}}}
\newcommand{\ExtensionTok}[1]{#1}
\newcommand{\FloatTok}[1]{\textcolor[rgb]{0.00,0.00,0.81}{#1}}
\newcommand{\FunctionTok}[1]{\textcolor[rgb]{0.00,0.00,0.00}{#1}}
\newcommand{\ImportTok}[1]{#1}
\newcommand{\InformationTok}[1]{\textcolor[rgb]{0.56,0.35,0.01}{\textbf{\textit{#1}}}}
\newcommand{\KeywordTok}[1]{\textcolor[rgb]{0.13,0.29,0.53}{\textbf{#1}}}
\newcommand{\NormalTok}[1]{#1}
\newcommand{\OperatorTok}[1]{\textcolor[rgb]{0.81,0.36,0.00}{\textbf{#1}}}
\newcommand{\OtherTok}[1]{\textcolor[rgb]{0.56,0.35,0.01}{#1}}
\newcommand{\PreprocessorTok}[1]{\textcolor[rgb]{0.56,0.35,0.01}{\textit{#1}}}
\newcommand{\RegionMarkerTok}[1]{#1}
\newcommand{\SpecialCharTok}[1]{\textcolor[rgb]{0.00,0.00,0.00}{#1}}
\newcommand{\SpecialStringTok}[1]{\textcolor[rgb]{0.31,0.60,0.02}{#1}}
\newcommand{\StringTok}[1]{\textcolor[rgb]{0.31,0.60,0.02}{#1}}
\newcommand{\VariableTok}[1]{\textcolor[rgb]{0.00,0.00,0.00}{#1}}
\newcommand{\VerbatimStringTok}[1]{\textcolor[rgb]{0.31,0.60,0.02}{#1}}
\newcommand{\WarningTok}[1]{\textcolor[rgb]{0.56,0.35,0.01}{\textbf{\textit{#1}}}}
\usepackage{graphicx,grffile}
\makeatletter
\def\maxwidth{\ifdim\Gin@nat@width>\linewidth\linewidth\else\Gin@nat@width\fi}
\def\maxheight{\ifdim\Gin@nat@height>\textheight\textheight\else\Gin@nat@height\fi}
\makeatother
% Scale images if necessary, so that they will not overflow the page
% margins by default, and it is still possible to overwrite the defaults
% using explicit options in \includegraphics[width, height, ...]{}
\setkeys{Gin}{width=\maxwidth,height=\maxheight,keepaspectratio}
\IfFileExists{parskip.sty}{%
\usepackage{parskip}
}{% else
\setlength{\parindent}{0pt}
\setlength{\parskip}{6pt plus 2pt minus 1pt}
}
\setlength{\emergencystretch}{3em}  % prevent overfull lines
\providecommand{\tightlist}{%
  \setlength{\itemsep}{0pt}\setlength{\parskip}{0pt}}
\setcounter{secnumdepth}{0}
% Redefines (sub)paragraphs to behave more like sections
\ifx\paragraph\undefined\else
\let\oldparagraph\paragraph
\renewcommand{\paragraph}[1]{\oldparagraph{#1}\mbox{}}
\fi
\ifx\subparagraph\undefined\else
\let\oldsubparagraph\subparagraph
\renewcommand{\subparagraph}[1]{\oldsubparagraph{#1}\mbox{}}
\fi

%%% Use protect on footnotes to avoid problems with footnotes in titles
\let\rmarkdownfootnote\footnote%
\def\footnote{\protect\rmarkdownfootnote}

%%% Change title format to be more compact
\usepackage{titling}

% Create subtitle command for use in maketitle
\providecommand{\subtitle}[1]{
  \posttitle{
    \begin{center}\large#1\end{center}
    }
}

\setlength{\droptitle}{-2em}

  \title{Data-tidying-lesson}
    \pretitle{\vspace{\droptitle}\centering\huge}
  \posttitle{\par}
    \author{Amanda B. Young}
    \preauthor{\centering\large\emph}
  \postauthor{\par}
      \predate{\centering\large\emph}
  \postdate{\par}
    \date{10/9/2019}


\begin{document}
\maketitle

\begin{Shaded}
\begin{Highlighting}[]
\KeywordTok{library}\NormalTok{(tidyr)}
\KeywordTok{library}\NormalTok{(ggplot2)}
\KeywordTok{library}\NormalTok{(dplyr)}
\end{Highlighting}
\end{Shaded}

To access \texttt{filter} from the \texttt{stats} package:
\texttt{stats::filter()}

\#Read in and clean up data

\begin{Shaded}
\begin{Highlighting}[]
\NormalTok{catch_original <-}\StringTok{ }\KeywordTok{read.csv}\NormalTok{(}\KeywordTok{url}\NormalTok{(}\StringTok{"https://knb.ecoinformatics.org/knb/d1/mn/v2/object/df35b.302.1"}\NormalTok{, }\DataTypeTok{method =} \StringTok{"libcurl"}\NormalTok{),}
                    \DataTypeTok{stringsAsFactors =} \OtherTok{FALSE}\NormalTok{)}
\KeywordTok{head}\NormalTok{(catch_original)}
\end{Highlighting}
\end{Shaded}

\begin{verbatim}
##   Region Year Chinook Sockeye Coho Pink Chum All notesRegCode
## 1    SSE 1886       0       5    0    0    0   5             
## 2    SSE 1887       0     155    0    0    0 155             
## 3    SSE 1888       0     224   16    0    0 240             
## 4    SSE 1889       0     182   11   92    0 285             
## 5    SSE 1890       0     251   42    0    0 292             
## 6    SSE 1891       0     274   24    0    0 298
\end{verbatim}

\texttt{stringsAsFactors} reads characters as factors rather than
characters, keeps the future issues down Try to allways set
\texttt{stingsAsFactors} to \texttt{FALSE}

Select the columns we want using the \texttt{select()}

\begin{Shaded}
\begin{Highlighting}[]
\NormalTok{catch_data<-catch_original }\OperatorTok\StringTok{ }
\StringTok{  }\CommentTok{#select(Region, Year, Chinook, Sockeye, Coho, Pink, Chum) }
\StringTok{  }\KeywordTok{select}\NormalTok{(}\OperatorTok{-}\NormalTok{All, }\OperatorTok{-}\NormalTok{notesRegCode)}
\KeywordTok{head}\NormalTok{(catch_data)}
\end{Highlighting}
\end{Shaded}

\begin{verbatim}
##   Region Year Chinook Sockeye Coho Pink Chum
## 1    SSE 1886       0       5    0    0    0
## 2    SSE 1887       0     155    0    0    0
## 3    SSE 1888       0     224   16    0    0
## 4    SSE 1889       0     182   11   92    0
## 5    SSE 1890       0     251   42    0    0
## 6    SSE 1891       0     274   24    0    0
\end{verbatim}

\begin{Shaded}
\begin{Highlighting}[]
\KeywordTok{summary}\NormalTok{(catch_data)}
\end{Highlighting}
\end{Shaded}

\begin{verbatim}
##     Region               Year        Chinook             Sockeye        
##  Length:1708        Min.   :1878   Length:1708        Min.   :    0.00  
##  Class :character   1st Qu.:1922   Class :character   1st Qu.:    6.75  
##  Mode  :character   Median :1947   Mode  :character   Median :  330.50  
##                     Mean   :1946                      Mean   : 1401.09  
##                     3rd Qu.:1972                      3rd Qu.:  995.50  
##                     Max.   :1997                      Max.   :44269.00  
##       Coho             Pink              Chum        
##  Min.   :   0.0   Min.   :    0.0   Min.   :    0.0  
##  1st Qu.:   0.0   1st Qu.:    0.0   1st Qu.:    0.0  
##  Median :  41.5   Median :   34.5   Median :   63.0  
##  Mean   : 150.4   Mean   : 2357.8   Mean   :  422.0  
##  3rd Qu.: 175.0   3rd Qu.: 1622.5   3rd Qu.:  507.5  
##  Max.   :3220.0   Max.   :53676.0   Max.   :10459.0
\end{verbatim}

Change the values in the CHinook column to numeric using
\texttt{mutate()}

\begin{Shaded}
\begin{Highlighting}[]
\NormalTok{catch_clean<-}\StringTok{ }\NormalTok{catch_data }\OperatorTok\StringTok{ }
\StringTok{  }\KeywordTok{mutate}\NormalTok{(}\DataTypeTok{Chinook=}\KeywordTok{as.numeric}\NormalTok{(Chinook))}
\end{Highlighting}
\end{Shaded}

\begin{verbatim}
## Warning: NAs introduced by coercion
\end{verbatim}

\begin{Shaded}
\begin{Highlighting}[]
\KeywordTok{head}\NormalTok{(catch_clean)}
\end{Highlighting}
\end{Shaded}

\begin{verbatim}
##   Region Year Chinook Sockeye Coho Pink Chum
## 1    SSE 1886       0       5    0    0    0
## 2    SSE 1887       0     155    0    0    0
## 3    SSE 1888       0     224   16    0    0
## 4    SSE 1889       0     182   11   92    0
## 5    SSE 1890       0     251   42    0    0
## 6    SSE 1891       0     274   24    0    0
\end{verbatim}

Investigate using \texttt{which()} and \texttt{is.na()}

\begin{Shaded}
\begin{Highlighting}[]
\NormalTok{i<-}\StringTok{ }\KeywordTok{which}\NormalTok{(}\KeywordTok{is.na}\NormalTok{(catch_clean}\OperatorTok{$}\NormalTok{Chinook))}
\NormalTok{i}
\end{Highlighting}
\end{Shaded}

\begin{verbatim}
## [1] 401
\end{verbatim}

\begin{Shaded}
\begin{Highlighting}[]
\NormalTok{catch_original[i,]}
\end{Highlighting}
\end{Shaded}

\begin{verbatim}
##     Region Year Chinook Sockeye Coho Pink Chum All notesRegCode
## 401    GSE 1955       I      66    0    0    1  68
\end{verbatim}

\begin{itemize}
\tightlist
\item
  use \texttt{mutate} to change the I to 1
\item
  use \texttt{mutate} to coerce Chinook column to numeric
  \texttt{\%in\%} can be used instead of \texttt{\textbar{}} as an or
  when using ifelse
\end{itemize}

\begin{Shaded}
\begin{Highlighting}[]
\NormalTok{catch_clean<-catch_data }\OperatorTok\StringTok{ }
\StringTok{  }\CommentTok{#mutate(Chinook = ifelse(Chinook == "I"| Chinook =="l")) %>% }
\StringTok{  }\KeywordTok{mutate}\NormalTok{(}\DataTypeTok{Chinook =} \KeywordTok{ifelse}\NormalTok{(Chinook  }\OperatorTok\StringTok{  }\KeywordTok{c}\NormalTok{(}\StringTok{"I"}\NormalTok{, }\StringTok{"l"}\NormalTok{), }\DecValTok{1}\NormalTok{, Chinook)) }\OperatorTok\StringTok{ }
\StringTok{  }\KeywordTok{mutate}\NormalTok{(}\DataTypeTok{Chinook=}\KeywordTok{as.numeric}\NormalTok{(Chinook))}
\KeywordTok{head}\NormalTok{(catch_clean)}
\end{Highlighting}
\end{Shaded}

\begin{verbatim}
##   Region Year Chinook Sockeye Coho Pink Chum
## 1    SSE 1886       0       5    0    0    0
## 2    SSE 1887       0     155    0    0    0
## 3    SSE 1888       0     224   16    0    0
## 4    SSE 1889       0     182   11   92    0
## 5    SSE 1890       0     251   42    0    0
## 6    SSE 1891       0     274   24    0    0
\end{verbatim}

\hypertarget{tidy-data}{%
\section{Tidy Data}\label{tidy-data}}

Move from wide formate to tall format using \texttt{pivot\_longer}

\begin{Shaded}
\begin{Highlighting}[]
\NormalTok{catch_long<-catch_clean }\OperatorTok\StringTok{ }
\StringTok{  }\KeywordTok{pivot_longer}\NormalTok{(}\DataTypeTok{cols=}\OperatorTok{-}\KeywordTok{c}\NormalTok{(Region, Year), }
               \DataTypeTok{names_to =} \StringTok{"Species"}\NormalTok{,}
               \DataTypeTok{values_to =} \StringTok{"Catch"}\NormalTok{)}
\KeywordTok{head}\NormalTok{(catch_long)}
\end{Highlighting}
\end{Shaded}

\begin{verbatim}
## # A tibble: 6 x 4
##   Region  Year Species Catch
##   <chr>  <int> <chr>   <dbl>
## 1 SSE     1886 Chinook     0
## 2 SSE     1886 Sockeye     5
## 3 SSE     1886 Coho        0
## 4 SSE     1886 Pink        0
## 5 SSE     1886 Chum        0
## 6 SSE     1887 Chinook     0
\end{verbatim}

\begin{Shaded}
\begin{Highlighting}[]
\NormalTok{catch_wide<-catch_long }\OperatorTok\StringTok{ }
\StringTok{  }\KeywordTok{pivot_wider}\NormalTok{(}\DataTypeTok{names_from =}\NormalTok{ Species,}
              \DataTypeTok{values_from =}\NormalTok{ Catch)}
\KeywordTok{head}\NormalTok{(catch_wide)}
\end{Highlighting}
\end{Shaded}

\begin{verbatim}
## # A tibble: 6 x 7
##   Region  Year Chinook Sockeye  Coho  Pink  Chum
##   <chr>  <int>   <dbl>   <dbl> <dbl> <dbl> <dbl>
## 1 SSE     1886       0       5     0     0     0
## 2 SSE     1887       0     155     0     0     0
## 3 SSE     1888       0     224    16     0     0
## 4 SSE     1889       0     182    11    92     0
## 5 SSE     1890       0     251    42     0     0
## 6 SSE     1891       0     274    24     0     0
\end{verbatim}

\texttt{rename()} catch to catch\_thousands \texttt{mutate()} to

\begin{Shaded}
\begin{Highlighting}[]
\NormalTok{catch_long<-catch_long }\OperatorTok\StringTok{ }
\StringTok{  }\KeywordTok{mutate}\NormalTok{(}\DataTypeTok{Catch=}\NormalTok{Catch}\OperatorTok{*}\DecValTok{1000}\NormalTok{) }
\end{Highlighting}
\end{Shaded}

\hypertarget{summarize-data}{%
\section{Summarize Data}\label{summarize-data}}

mean catch by region of Sockeye

\begin{Shaded}
\begin{Highlighting}[]
\NormalTok{mean_region<-catch_long }\OperatorTok\StringTok{ }
\StringTok{  }\KeywordTok{group_by}\NormalTok{(Region) }\OperatorTok\StringTok{ }
\StringTok{    }\KeywordTok{filter}\NormalTok{(Species }\OperatorTok{==}\StringTok{ "Sockeye"}\NormalTok{) }\OperatorTok\StringTok{ }
\KeywordTok{summarise}\NormalTok{(}\DataTypeTok{mean_catch=}\KeywordTok{mean}\NormalTok{(Catch),}
            \DataTypeTok{n_obs=}\KeywordTok{n}\NormalTok{())}
\KeywordTok{head}\NormalTok{(mean_region)}
\end{Highlighting}
\end{Shaded}

\begin{verbatim}
## # A tibble: 6 x 3
##   Region mean_catch n_obs
##   <chr>       <dbl> <int>
## 1 ALU         6839.    87
## 2 BER        36882.   102
## 3 BRB     12516500    114
## 4 CHG       962600    110
## 5 CKI      1772952.   105
## 6 COP       699628.    94
\end{verbatim}

total annual catch

\begin{Shaded}
\begin{Highlighting}[]
\NormalTok{annual_catch<-catch_long }\OperatorTok\StringTok{ }
\StringTok{  }\KeywordTok{group_by}\NormalTok{(Year) }\OperatorTok\StringTok{ }
\StringTok{  }\KeywordTok{summarise}\NormalTok{(}\DataTypeTok{total_catch =} \KeywordTok{sum}\NormalTok{(Catch)) }\OperatorTok\StringTok{ }
\StringTok{  }\KeywordTok{arrange}\NormalTok{(}\KeywordTok{desc}\NormalTok{(total_catch))}
\NormalTok{annual_catch}
\end{Highlighting}
\end{Shaded}

\begin{verbatim}
## # A tibble: 120 x 2
##     Year total_catch
##    <int>       <dbl>
##  1  1995   217260000
##  2  1994   195882000
##  3  1993   193107000
##  4  1991   189520000
##  5  1996   176072000
##  6  1990   155058000
##  7  1989   154127000
##  8  1985   146359000
##  9  1992   136726000
## 10  1984   133964000
## # ... with 110 more rows
\end{verbatim}

\begin{Shaded}
\begin{Highlighting}[]
\KeywordTok{ggplot}\NormalTok{(annual_catch, }\KeywordTok{aes}\NormalTok{(}\DataTypeTok{x=}\NormalTok{Year, }\DataTypeTok{y=}\NormalTok{total_catch))}\OperatorTok{+}
\StringTok{  }\KeywordTok{geom_point}\NormalTok{()}
\end{Highlighting}
\end{Shaded}

\includegraphics{Tidy_RM_files/figure-latex/unnamed-chunk-3-1.pdf}

\begin{Shaded}
\begin{Highlighting}[]
\NormalTok{catch_long }\OperatorTok\StringTok{ }
\StringTok{  }\KeywordTok{group_by}\NormalTok{(Year,Species) }\OperatorTok\StringTok{ }
\StringTok{  }\KeywordTok{summarise}\NormalTok{(}\DataTypeTok{total_catch =} \KeywordTok{sum}\NormalTok{(Catch)) }\OperatorTok\StringTok{ }
\StringTok{  }\KeywordTok{arrange}\NormalTok{(}\KeywordTok{desc}\NormalTok{(total_catch)) }\OperatorTok\StringTok{ }
\StringTok{  }\KeywordTok{ggplot}\NormalTok{(}\KeywordTok{aes}\NormalTok{(}\DataTypeTok{x=}\NormalTok{Year, }\DataTypeTok{y=}\NormalTok{total_catch, }\DataTypeTok{colour=}\NormalTok{Species))}\OperatorTok{+}
\StringTok{  }\KeywordTok{geom_point}\NormalTok{()}
\end{Highlighting}
\end{Shaded}

\includegraphics{Tidy_RM_files/figure-latex/unnamed-chunk-4-1.pdf} )

\begin{Shaded}
\begin{Highlighting}[]
\NormalTok{region_defs<-}\StringTok{ }\KeywordTok{read.csv}\NormalTok{(}\KeywordTok{url}\NormalTok{(}\StringTok{"https://knb.ecoinformatics.org/knb/d1/mn/v2/object/df35b.303.1"}\NormalTok{, }\DataTypeTok{method =} \StringTok{"libcurl"}\NormalTok{),}
                    \DataTypeTok{stringsAsFactors =} \OtherTok{FALSE}\NormalTok{) }\OperatorTok\StringTok{ }
\StringTok{  }\KeywordTok{select}\NormalTok{(code, mgmtArea) }\OperatorTok\StringTok{ }
\StringTok{  }\KeywordTok{rename}\NormalTok{(}\DataTypeTok{Region =}\NormalTok{ code)}
\KeywordTok{head}\NormalTok{(region_defs)}
\end{Highlighting}
\end{Shaded}

\begin{verbatim}
##    Region                                  mgmtArea
## 1     GSE              Unallocated Southeast Alaska
## 2     NSE                 Northern Southeast Alaska
## 3     SSE                 Southern Southeast Alaska
## 4     YAK                                   Yakutat
## 5 PWSmgmt      Prince William Sound Management Area
## 6     BER Bering River Subarea Copper River Subarea
\end{verbatim}

Create a join between catch and region files

\begin{Shaded}
\begin{Highlighting}[]
\CommentTok{#catch_joined<-left_join(catch_long, region_defs, by=c("Region"= "code"))}
\NormalTok{catch_joined<-}\KeywordTok{left_join}\NormalTok{(catch_long, region_defs, }\DataTypeTok{by=}\StringTok{"Region"}\NormalTok{)}
\NormalTok{catch_joined<-}\StringTok{ }\KeywordTok{left_join}\NormalTok{(catch_long, region_defs)}
\end{Highlighting}
\end{Shaded}

\begin{verbatim}
## Joining, by = "Region"
\end{verbatim}

\begin{Shaded}
\begin{Highlighting}[]
\KeywordTok{head}\NormalTok{(catch_joined)}
\end{Highlighting}
\end{Shaded}

\begin{verbatim}
## # A tibble: 6 x 5
##   Region  Year Species Catch mgmtArea                 
##   <chr>  <int> <chr>   <dbl> <chr>                    
## 1 SSE     1886 Chinook     0 Southern Southeast Alaska
## 2 SSE     1886 Sockeye  5000 Southern Southeast Alaska
## 3 SSE     1886 Coho        0 Southern Southeast Alaska
## 4 SSE     1886 Pink        0 Southern Southeast Alaska
## 5 SSE     1886 Chum        0 Southern Southeast Alaska
## 6 SSE     1887 Chinook     0 Southern Southeast Alaska
\end{verbatim}

\begin{Shaded}
\begin{Highlighting}[]
\NormalTok{dates_df <-}\StringTok{ }\KeywordTok{data.frame}\NormalTok{(}\DataTypeTok{date =} \KeywordTok{c}\NormalTok{(}\StringTok{"5/24/1930"}\NormalTok{,}
                                \StringTok{"5/25/1930"}\NormalTok{,}
                                \StringTok{"5/26/1930"}\NormalTok{,}
                                \StringTok{"5/27/1930"}\NormalTok{,}
                                \StringTok{"5/28/1930"}\NormalTok{),}
                       \DataTypeTok{stringsAsFactors =} \OtherTok{FALSE}\NormalTok{)}

\NormalTok{dates_df }\OperatorTok\StringTok{ }
\StringTok{  }\KeywordTok{separate}\NormalTok{(date, }\DataTypeTok{into =} \KeywordTok{c}\NormalTok{(}\StringTok{"month"}\NormalTok{, }\StringTok{"day"}\NormalTok{, }\StringTok{"year"}\NormalTok{), }\DataTypeTok{sep =} \StringTok{"/"}\NormalTok{, }\DataTypeTok{remove =} \OtherTok{FALSE}\NormalTok{) }\OperatorTok\StringTok{ }
\StringTok{  }\KeywordTok{unite}\NormalTok{(date_}\DecValTok{2}\NormalTok{, year, month, day, }\DataTypeTok{sep=}\StringTok{"-"}\NormalTok{)}
\end{Highlighting}
\end{Shaded}

\begin{verbatim}
##        date    date_2
## 1 5/24/1930 1930-5-24
## 2 5/25/1930 1930-5-25
## 3 5/26/1930 1930-5-26
## 4 5/27/1930 1930-5-27
## 5 5/28/1930 1930-5-28
\end{verbatim}

can use substr to seperate a variable by the number of integers, also
can use package \texttt{stringr::str\_sub} to get the last few integers
\texttt{string\_pad} can say I want two digits but if I only have a 5
then it would become 05


\end{document}
